\pdfbookmark{Abstract}{Abstract}

\begin{abstract}
With the evolution of social media platforms, the amount of unlabeled information has gone skyrocketing. The process of labeling this kind of information is evermore complex. Typical approaches used on the WEB for Topic Detection and Tracking cannot be directly applied due to the small amount of text produced per tweet, orthographic errors, abbreviations and so on.

In this thesis, we propose and analyze a new form of topic detection and tracking on social networks. By leveraging the social relations between authors of the gathered content, and apply them to the clustering process.

In order to achieve this, we proposed some modifications to the artificial neural network  and clustering algorithm --- Self Organizing Maps. 
\end{abstract}

\begin{keywords}
topic detection, twitter, self-organizing maps, classification, clustering
\end{keywords}
\clearpage
\thispagestyle{empty}
\cleardoublepage

\pdfbookmark{Resumo}{Resumo}
\begin{resumo}
 Com a evolução das plataformas sociais, ocorreu um aumento na quantidade de informação não categorizada. No entanto, o processo de categorização deste tipo de informação é bastante complexo, uma vez que os métodos mais comuns de Detecção e Rastreamento de Tópicos na WEB não são directamente aplicáveis em dados criados nas redes sociais. Isto porque o texto produzido por cada tweet é demasiado pequeno, tem muitos erros ortográficos e abreviações, etc.

Nesta tese é proposta e analizada uma nova forma de detecção e rastreamento de tópicos em redes sociais, onde as relações sociais entre os autores do conteúdo foram tidas em conta no processo de clustering.

Para atingir este objectivo, propuseram-se algumas modificações aos Mapas de Kohonen --- um algoritmo de clustering e rede neuronal artificial.
\end{resumo}

\begin{palavraschave}
  detecção de tópicos, twitter, mapas auto organizados, classificação, agrupamento
\end{palavraschave}

\clearpage
\thispagestyle{empty}
\cleardoublepage

% Required for the fancy chapters
\dominitoc
\dominilof
\dominilot
 
