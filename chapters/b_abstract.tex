\pdfbookmark{Abstract}{Abstract}

\begin{abstract}
 %TODO ADD: social data has very few text but a lot hidden signals 
 %          tweets in particullar have even few text, makes them harder to cathegorize
  Clustering is a widely used technique in data analysis. In this thesis, a generically \acrodef{ANN}[artificial neural network] algorithm used for clustering is modified in order to enhance the value of socially connected ententies.

To achieve this, we present RubySOM. A framework for easy construction of custom Self-Organizing Maps. With it, it is possible to dinammically change multiple parts of the algorithm, making it extremlly flexible solution to create, train and run custom implementations of the algorithm. 

With RubySOM, a relational aware version of the SOM algorithm was created in order to better identify topics on the social network twitter. 
\end{abstract}

\begin{keywords}
topic detection, twitter, self-organizing maps, classification, clustering
\end{keywords}
\clearpage
\thispagestyle{empty}
\cleardoublepage

\pdfbookmark{Resumo}{Resumo}
\begin{resumo}

\end{resumo}

\begin{palavraschave}
  detecção de tópicos, twitter, mapas auto organizados, classificação, agrupamento
\end{palavraschave}

\clearpage
\thispagestyle{empty}
\cleardoublepage

% Required for the fancy chapters
\dominitoc
\dominilof
\dominilot
 
