\fancychapter{State of the art}
The second chapter can be the state of the art for the work you are preenting. Sometimes it appears under the previous chapter -- Introduction. Though I typically prefer to put it in a separate chapter.

In most cases this chapter addresses some important technology or knowledge. Name this chapter according to the topic.

The introduction should be written assuming that the reader has little or no knowledge of your specific problem. Try to write this chapter in a clear manner that makes it easier for the reader to understand the remaining of your thesis. Append information to this chapter as needed when writting other chapters.

Do not forget to add references to your document. In general, if you are stating something which is not obvious nor it is of common knowledge amongst the scientific or engineering community, you should place a reference. To ease the process of referencing I include a couple of references to my own papers herein. Here are some examples: \cite{tomas2010bda}, \citep{tomas2010aqa}, \citet{ramalho2010eic}, \mbox{\cite[(2.2)]{tomas2009nct}}.

To help you make references, you can try the commands or consult the manual. The following link can also be a good help:
\mbox{\underline{http://merkel.zoneo.net/Latex/natbib.php}}.

As a side note, remember that you should not include links in your dissertation as I just made. Instead you should put a reference as follows: \cite{mylink}. Remember that in this case, the reference date is the date you have last consulted the page.

\section{Summary}

It is typical a good ideia to have an ending section summarizing the chapter.

% Ensure that the next chapter starts in a odd page
\cleardoublepage 
