\fancychapter{Conclusions and Future Work}
\label{ch:conclusions}
This thesis main goal was to find topics on social networks, using clustering techniques. Due to overgrowing amount of pertinent information on social networks, the process of labelling such kind of data has been proven a huge challenge in modern \ac{IR} techniques. 
With this in mind, we started by analysing the state of the art techniques for \ac{TDT} used on Twitter. Researched how \ac{SOM} are used in specific contexts, in order to better hunderstand what adaptations could be applyed to the \ac{SOM} algorithm, in order for it to better cathegorize data from social networks.

Afterwards we outlined how it would be possible to use \ac{SOM} with twitter data, and specified how the neural network algorithm could be changed in order to better manage socially connected data. 

In order to create this new version of the \ac{SOM}, first a \ac{SOM} framework was built and tested. Only then, the algorithm of the default \ac{SOM} was changed in order to create the Homophilic SOM.


% Ensure that the next chapter starts in a odd page
\cleardoublepage
 
