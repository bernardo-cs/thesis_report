\fancychapter{Evaluation Metrics}


\section{Conclusions}

% Ensure that the next chapter starts in a odd page
\cleardoublepage
 
%%%%%% REMOVED STUFF
%\subsection{Topology Preservation} 
%\label{sub:topology_preservation}
%The Self-Organizing Map performs a mapping from the n-dimensional input space into the two dimensional output space and where resides one the most fascinating characteristics, which is that the output map tries to preserve the topology from the input space. This grants the SOM algorithm a way to visualize high-dimensional data that other neural networks or clustering algorithms don't have. Even though this is true, sometimes during training it is not possible to preserve the topology of the network.
%Thus topology preservation can be measured through the Topographic error~\citet{Kiviluoto1996} which is the proportion of all data vectors for which first and second BMUs \footnote{unit that is closest to the winning neuron. BMU Best fitting unit } are not adjacent units.
%In this project the Topographic Error will be calculated for all SOM implementations and VSM usages in order to understand if the representation of the SOM output space is well defined.

 %\begin{itemize}
  %\item show UMatrixes and multiple steps map trainning of the SOM library trainnig
  %\item show metrics for the crawller, tweets per second, users persecond, size of the dump a long the time.
  %\item compare my som library with other som libraries: training velocity with diferent parameters, map after trainned.
  %\item Compare Homophilic-SOM results with non homophilic: UMatrixes, cluster results, Quantization error, jacknife. 
%\end{itemize}
%%\section{Evaluation Metrics} 
%%\label{sec:evaluation_metrics}
%Evaluation of the topic detection on Tweets will be made in two distinct ways. The first way will focus on  binary classification using the precision and recall metrics, and will be described in Subsection~\ref{sub:testing_for_precision_and_recall}. The second way will focus on statistically testing the SOM learning process and the computed trained network. This testing process will be described in Subsection~\ref{sub:cluster_quality_testing}. 

%\section{Testing for Precision and Recall} 
%\label{sec:testing_for_precision_and_recall}
%Precision and Recall are both ways to measure the rate of right guesses made by the trained SOM network, and are defined in the following way:
%\begin{itemize}
  %\item \textbf{Precision:} Fraction of retrieved instances that where relevant 
    %\begin{equation}
  precision = \frac{|{relevant\;documents}\cap{retrieved\;documents}|}{{retrieved\;documents}}
\end{equation} 

  %\item \textbf{Recall:} Fraction of relevant instances that where retrieved
    %\begin{equation}
  recall = \frac{|{relevant\;documents}\cap{retrieved\;documents}|}{{relevant\;documents}} 
\end{equation} 

%\end{itemize}
 
%In order to calculate Precision and Recall we need to have the \emph{relevant documents} and the \emph{retrieved documents}. The \emph{relevant documents} are rather hard to determine because they need to be categorized by humans, which is an expensive task.
