\fancychapter{Introduction}
Write your introduction in here. You should start by introducing your thesis in a way that the reader understands where your thesis stand in todays technology and overall knowledge. Consider answering questions such as:
\begin{enumerate}
 \item what is the background of your work?
 \item how do your work fit in todays knowledge? Have you done something that does not exist in the market? What are the differences? What is missing in nowadays products and solutions?
 \item is your thesis made as part of a larger project? If so, describe it.
 \item is your thesis usefull for some work environment? If so, describe it.
\end{enumerate}

\section{Motivation}

Given the general description provided previously, what is the motivation of your work. Explain why the product or solution developed in the course of your thesis is important.

\section{Objectives}

This section should describe the objectives of your work.

\section{Main contributions}
Do not forget this one. Notice that the objectives is what you have proposed to do. Main contributions are the innovations of your work, or, in other cases, what your work is really good at. If you submitted/published an article in a peer-reviewed conference or journal, do not forget to state here.

\section{Dissertation outline}

Explain how did you organized your thesis.

% If you prefer to have multiple files, each storing, for example, a different chapter (or a section within a chapter)
% you can use one of the following commands:
%
% \input{my_tex_file.tex}
% \include{my_tex_file.tex}
% 
% the difference between the two commands is that \include will always start a new page, whereas \input does not

\cleardoublepage 
